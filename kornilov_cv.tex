\documentclass[9pt]{article}

\usepackage[T2A]{fontenc}     % внутренняя T2A кодировка TeX
\usepackage[utf8]{inputenc}
\usepackage[margin=0.5in]{geometry}
\pagestyle{empty} % нумерация выкл.
\addtolength{\textheight}{1.75in}
\usepackage{hyperref}
\usepackage{longtable}
\usepackage{color}
\usepackage{setspace}
\usepackage{multicol}
\setlength{\columnsep}{30pt}
\usepackage{fontawesome}
\setlength\parindent{0em}
\usepackage{xcolor}

\definecolor{gray}{rgb}{0.4,0.4,0.4}
\definecolor{mylinkcolor}{HTML}{2200CC}

\hypersetup{
    colorlinks=true,
    urlcolor=mylinkcolor
}

\begin{document}

\begin{multicols}{2}

\noindent
\vspace{3em}
\noindent

{\huge{\textbf{Maxim Kornilov}}}

Computer science student

Potsdam, Germany
\vspace{1.0em}

\href{https://github.com/kornilov-mr}{github.com/kornilov-mr}

\href{https://t.me/Gatehost}{t.me/Gatehost}

kornilov.mr@gmail.com

\vspace{5em}

{\textbf{Skills}}
\vspace{0.5em}
\hrule
\vspace{1.0em}

Java, C\#, SQL, Spring, Java Swing, Rest, Jakarta, C, GPU, Python.

\vspace{1em}
{\textbf{Personal Projects}}
\vspace{0.5em}
\hrule
\vspace{1em}

{\textbf{Polygon 3D editor}}\\[1\baselineskip]
\href{https://github.com/kornilov-mr/PolygonMesh}{github.com/kornilov-mr/PolygonMesh}

Ui application for editing Polygon 3d models written from scrutch

\vspace{2em}
{\textbf{Education}}
\vspace{0.5em}
\hrule
\vspace{1em}

{\textbf{Potsdam University}} \color{gray} Potsdam, Germany \color{black}
\\[1\baselineskip]
\color{black} First course \color{black}
\vspace{0.3em}
\color{black} Informatik/Computer science \color{black}

\columnbreak

\vspace{0.5em}

{\textbf{Experience}}

\vspace{10.5em}




\end{multicols}

\end{document}             % End of document.     
